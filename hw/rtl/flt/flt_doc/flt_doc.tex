\documentclass[letterpaper, 12pt]{article}

\usepackage{graphicx}
\usepackage{mathtools}

\begin{document}

%Page titre

\section{Conventions}

\subsection{Variables}
$s$ : Valeur du bit de signe\\
$e$ : Valeur binaire de l'exposant\\
$m$ : Valeur binaire de la mantisse\\
$w$ : Nombre de bits de l'exposant\\
$t$ : Nombre de bits de la mantisse exculant le LSB\\
$p$ : Nombre de bits de la mantisse incluant le LSB ommit\\
$E$ : Valeur de l’exposant\\
$M$ : Valeur de la mantisse\\
$b$ : biais de l'exposant\\

\subsection{Équations}

$$b = 2^{w-1}-1$$
$$E = e – b$$
$$Emin = 2-2^{w-1}$$
$$Emax = 2^{w-1}-1$$

$$Inf : E=2^{w-1} \& m=0$$
$$qNaN : E=2^{w-1} \& m/=0 \& m[-1]=0$$
$$sNaN : E=2^{w-1} \& m/=0 \& m[-1]=1$$

$$Subnormal: E=0$$

$$Normal : (-1)^s \cdot 2^E \cdot (1 + m \cdot 2^{-t})$$
$$Subnormal : (-1)^s \cdot 2^{Emin} \cdot m \cdot 2^{-t}$$

\section{FMUL}

Le multiplicateur exécute la fonction $Y = A * B$

\subsection{Operation standard}
1. normal x normal = normal
$$ Y = A \times B $$
$$ (-1)^{s_y} \cdot 2^{E_y} \cdot (1 + m_y\cdot2^{-t}) = (-1)^{s_a} \cdot 2^{E_a} \cdot (1 + m_a \cdot 2^{-t}) \times (-1)^{s_b} \cdot 2^{E_b} \cdot (1 + m_b \cdot 2^{-t})$$
$$ (-1)^{s_y} \cdot 2^{E_y} \cdot (1 + m_y\cdot2^{-t}) = (-1)^{s_a+s_b} \cdot 2^{E_a+E_b} \cdot (1 + m_a \cdot 2^{-t}) \times (1 + m_b \cdot 2^{-t})$$

$$ s_y = s_a + s_b $$
$$ E_y = E_a + E_b $$
$$ (1 + m_y\cdot2^{-t}) = (1 + m_a \cdot 2^{-t}) \times (1 + m_b \cdot 2^{-t})$$

Agorithme exposant:
lorsque les 2 exp sont additionné, un bias en extra se trouve dans le resulat, on peut soustraire ce biais en effectuant -2**


2. normal x normal = subnormal

3. normal x subnormal = subnormal

4. normal x subnormal = normal

5. subnormal x subnormal = subnormal

6. subnormal x subnormal = normal
impossible
\subsection{Operation spéciales}
1.  NaN x NaN
La valeur de A est propagée

2. NaN x Valeur/0/InF
La valeur NaN est propagée

3. 0 x InF
NaN est généré

4. 0 x normal/subnormal
0 est généré

5. InF x InF
InF est généré

6. InF x valeur
InF est généré

%\chapter{FADD}
%
%\chapter{FMA}
%
%\chapter{FDIV}
%
%\chapter{SQRT}

\end{document}
